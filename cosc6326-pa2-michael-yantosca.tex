\documentclass[11pt,epsf]{article}
\usepackage{amssymb,amsmath,amsthm,amsfonts,mathrsfs,color}
\usepackage{epsfig}
\usepackage{latexsym}
\usepackage{verbatim}
\usepackage{setspace}
\usepackage{algorithm}
\usepackage[noend]{algorithmic}
\usepackage{algorithmicext}
\usepackage{ifthen}
\usepackage{graphicx}
\usepackage{url}
\usepackage{hyperref}
\usepackage[utf8]{luainputenc}
\usepackage[bibencoding=utf8,backend=biber]{biblatex}
\addbibresource{cosc6326-pa2-michael-yantosca.bib}
\usepackage{fancyhdr}
\pagestyle{fancy}
\lhead{{\footnotesize{COSC6326 PA 2}}}
\rhead{{\footnotesize{Michael Yantosca}}}

\usepackage{longtable}
\usepackage{pgfplots}
\usepackage{pgfplotstable}
\usepgfplotslibrary{external}
\usepgfplotslibrary{statistics}
\usepgfplotslibrary{groupplots}
\usetikzlibrary{pgfplots.groupplots, external}
\tikzexternalize[]
\pgfplotsset{
  tick label style={font=\footnotesize},
  label style={font=\small},
  legend style={font=\small},
  compat=newest
}
\pgfplotstableset{
  col sep=comma,
  begin table=\begin{longtable},
  end table=\end{longtable},
  every head row/.append style={after row=\endhead}
}

\newtheorem{fact}{Fact}
\newtheorem{theorem}{Theorem}
\newtheorem{lemma}{Lemma}
\newtheorem{claim}{Claim}
\newtheorem{remark}{Remark}
\newtheorem{definition}{Definition}
\newtheorem{corollary}{Corollary}
\newtheorem{proposition}{Proposition}
\newtheorem{example}{Example}
\newtheorem{observation}{Observation}
\newtheorem{exercise}{Exercise}
\newtheorem{statement}{Statement}
\newtheorem{problem}{Problem}

\newcommand{\TODO}[0]{\textbf{\color{red}{TODO}}}

% \linregplots{title}{prefix}{suffix}{x}{y}
\newcommand{\linregplots}[5]{
  \nextgroupplot[title=#1]
  \addplot [red, only marks, mark size=0.5] table [x=#4, y=#5] {#2.k1#3.log};
  \addplot [red, no markers] table [x=#4,y={create col/linear regression={y=#5}}] {#2.k1#3.log};
  \addplot [blue, only marks, mark size=0.5] table [x=#4, y=#5] {#2.k2#3.log};
  \addplot [blue, no markers] table [x=#4,y={create col/linear regression={y=#5}}] {#2.k2#3.log};
  \addplot [green, only marks, mark size=0.5] table [x=#4, y=#5] {#2.k4#3.log};
  \addplot [green, no markers] table [x=#4,y={create col/linear regression={y=#5}}] {#2.k4#3.log};
  \addplot [orange, only marks, mark size=0.5] table [x=#4, y=#5] {#2.k8#3.log};
  \addplot [orange, no markers] table [x=#4,y={create col/linear regression={y=#5}}] {#2.k8#3.log};
  \addplot [purple, only marks, mark size=0.5] table [x=#4, y=#5] {#2.k16#3.log};
  \addplot [purple, no markers] table [x=#4,y={create col/linear regression={y=#5}}] {#2.k16#3.log};
  \addplot [brown, only marks, mark size=0.5] table [x=#4, y=#5] {#2.k32#3.log};
  \addplot [brown, no markers] table [x=#4,y={create col/linear regression={y=#5}}] {#2.k32#3.log};
}

\date{}
\title{COSC6326 Programming Assignment 2}
\author{Michael Yantosca}
\begin{document}
\maketitle
\tableofcontents

\section{Introduction}{
  \paragraph{}{
    \TODO
  }
}

\section{Analysis}{
  \subsection{\texttt{text2mpig}}{
    \paragraph{}{
      \begin{algorithm}
        \footnotesize
        \caption{\textsc{Text-To-MPI-Graph}, Algorithm for Converting Edge-Centric Text Model to Compact Edge-Centric Form}
        \begin{algorithmic}
          \STATE{\TODO}
        \end{algorithmic}
      \end{algorithm}
    }
    \paragraph{}{
      \TODO
    }
  }

  \subsection{\texttt{genmpig}}{
    \paragraph{}{
      \begin{algorithm}
        \footnotesize
        \caption{\textsc{Generate-MPI-Graph}, Distributed Algorithm for Generating an Erdos-Renyi Graph Stored in Compact Edge-Centric Form}
        \begin{algorithmic}
          \STATE{\TODO}
        \end{algorithmic}
      \end{algorithm}
    }
    \paragraph{}{
      \TODO
    }
  }

  \subsection{\texttt{bfs-coco}}{
    \paragraph{}{
      \begin{algorithm}
        \footnotesize
        \caption{\textsc{BFS-Connected-Components}, Distributed BFS Algorithm for Determining Connected Components in an Arbitrary Graph}
        \label{alg:bfs-coco}
        \begin{algorithmic}
          \STATE{\TODO}
        \end{algorithmic}
      \end{algorithm}
    }
    \paragraph{}{
      \TODO
    }

    \begin{theorem}
      \label{thm:bfs-coco}
      Algorithm~\ref{alg:bfs-coco} determines the connected components in an arbitrary graph
      of $n$ vertices distributed over $k$ machines with communication complexity $O(???)$.
    \end{theorem}
    \begin{proof}
      Because reasons.
    \end{proof}
  }
}

\section{Results}{
  \subsection{Test Procedures}{
    \paragraph{}{
      Tests to validate correctness were performed locally on a dual-core laptop
      \footnote{2 x Intel(R) Core(TM) i5-6200U CPU @ 2.30GHz.} running Pop!OS.
      \footnote{An Ubuntu 18.04 variant.}
      Tests for which results were collected systematically and graphed were
      done on the UH \texttt{crill} cluster with the following parameters:
      \begin{itemize}
      \item{$n \in 2^{[10,20]}$, the total population size}
      \item{$k \in \{1,2,4,8,16,32\}$, the number of distributed nodes}
      \item{$\epsilon = 0.2$, the threshold error}
      \item{$p$, the existential probability of a given edge,
        \begin{align}
          p &
          \begin{cases}
            < \frac{(1 - \epsilon)\ln n}{n} \\
            = \frac{\ln n}{n} \\
            > \frac{(1 + \epsilon)\ln n}{n} \\
            < \frac{(1 - \epsilon)}{n} \\
            = \frac{1}{n} \\
            > \frac{(1 + \epsilon)}{n} \\
          \end{cases}
        \end{align}
      }
      \end{itemize}
      Each parameter combination was executed at least 3 times with
      several executables contributing to the testing process:
    }
    \paragraph{\texttt{txt2mpig}}{
      A utility program for converting SNAP edge-centric model text files
      into a compact edge-centric binary format suitable for accessing via MP/IO.
    }
    \paragraph{\texttt{genmpig}}{
      A utility program for generating Erdos-Renyi random graphs and saving
      in a compact edge-centric binary format.
    }
    \paragraph{\texttt{bfs-coco}}{
      A program which reads a compact edge-centric binary format file and
      distributes the vertex-centric equivalent across a $k$-machine context
      and subsequently executes Algorithm~\ref{alg:bfs-coco} on the distributed graph.
    }
    \paragraph{}{
      The reader is directed to the accompanying \texttt{README.md} for
      explicit usage instructions on the various programs as well as
      for a deeper explanation of implementation considerations,
      engineering tradeoffs, and known issues with the programs.
    }
    \paragraph{}{
      It should be noted that the MPI 2.1 standard\autocite{MPI21} and g++-7.2
      was used for development since they were available through the development
      laptop's package system, but the program compiled and ran on the \texttt{crill}
      cluster with g++-5.3.0 and MPI 3.0. The code requires the C++-11 standard.
    }
  }
}

\section{Conclusions}{
  \paragraph{}{
    \TODO
  }
}

\printbibliography
\end{document}
